\documentclass[11pt,letterpaper]{article}

\usepackage{setspace}
%\doublespacing
\linespread{1.25}
%\linespread{2}
\usepackage{geometry}

\geometry{letterpaper, margin=1in}
%\usepackage{times}
\usepackage{pslatex}
\usepackage{apacite}
\usepackage{url}
\usepackage{graphicx}
\usepackage{caption}
\usepackage{subcaption}
\usepackage{listings}
\usepackage{color}
\usepackage{textcomp}
\usepackage{amsmath}
\usepackage{amssymb}
\usepackage{wrapfig}
\usepackage{lipsum}

\graphicspath{{figures/}}

\def\signed #1{{\leavevmode\unskip\nobreak\hfil\penalty50\hskip2em
  \hbox{}\nobreak\hfil(#1)%
  \parfillskip=0pt \finalhyphendemerits=0 \endgraf}}

\newsavebox\mybox
\newenvironment{aquote}[1]
  {\savebox\mybox{#1}\begin{quote}}
  {\signed{\usebox\mybox}\end{quote}}


 \newcommand{\denote}[1]{\mbox{ $[\![ #1 ]\!]$}}

\definecolor{Red}{RGB}{255,0,0}
\newcommand{\red}[1]{\textcolor{Red}{#1}}  
\definecolor{Green}{RGB}{10,200,100}
\definecolor{Blue}{RGB}{10,100,200}
\newcommand{\ndg}[1]{\textcolor{Green}{[ndg: #1]}}  
\newcommand{\mht}[1]{\textcolor{Blue}{[mht: #1]}}  

\usepackage{titlesec}

\setcounter{secnumdepth}{4}

\titleformat{\paragraph}
{\normalfont\normalsize\bfseries}{\theparagraph}{1em}{}
\titlespacing*{\paragraph}
{0pt}{3.25ex plus 1ex minus .2ex}{1.5ex plus .2ex}

\title{Talking in generalizations}
\author{{\large \bf Michael Henry Tessler} and {\large \bf Noah D. Goodman} \\
\{mtessler, ngoodman\} @stanford.edu\\
  Department of Psychology, Stanford University}

\begin{document}

\maketitle

Learning and development require solving problems of induction.
These are hard problems though they can be solved by adults and young children with observational learning \cite{Markman1989} 
and in pedagogically enriched scenarios \emph{(CITE SOME PEDAGOGY}).
After about the age of the two, a new tool for communication and learning is available: language.

Generalizations are central to human understanding and thus, it is not surprising that language provides a simple and ubiquitous way to communicate these generalizations. 
Generic language (e.g.~\emph{Swans are white.}) and habitual language \emph{Bill smokes.}) convey generalizations about categories and events, respectively, and have similar semantic features that they are often described in the same breath \cite{Carlson1977, Carlson2005, Cohen1999}.
The primary semantic property that makes these sentences so similar to one another and at the same time, so different from the relatively well-understood quantifier sentences (e.g.~\emph{All swans are white.}) is in determine what makes generic or habitual sentences true or false (i.e. the truth conditions).
At first glance, generics seem like universally quantified statements as in \emph{All swans are white}, and yet generics---unlike universals---admit exceptions (e.g. there are black swans). 
Interpreting the generic as meaning ``most'' (i.e. \emph{Most swans are white}) captures many cases but fails to account for others: \emph{Robins lay eggs} seems true even though only adult, female robins do; only a tiny fraction of Mosquitos carry the virus Zika yet \emph{Mosquitos carry Zika} also sounds true. 
In fact, the very notion that the felicity of the generic can be tied to how many instances have the property (as the felicity of quantifier statements are) violates intuitions: \emph{Robins lay eggs} even though only the females do but \emph{Robins are female} (even though only the females are) is not a reasonable utterance.

Habitual sentences behave in an analogous way. 
Bill may smoke a pack a day and so we would say \emph{Bill smokes}, and if Bill goes without a cigarette for an entire family vacation (thus temporarily reducing his effective rate of smoking), we will not be so easily convinced (i.e. \emph{Bill smokes} is probably still true).
Cases like \emph{Mosquitos carry malaria} (wherein only a tiny percentage have the property) seem to parallel habitual sentences of rare actions like \emph{Susan writes novels}. Susan may only have written 3 novels in her life, but still this seems like a valid generalization to convey.


%Yet the meaning of generic language is philosophically puzzling and has resisted precise formalization.
%We explore the idea that the core meaning of a generic sentence is simple but underspecified, 
%and that general principles of pragmatic reasoning are responsible for establishing the precise meaning in context.
%Building on recent probabilistic models of language understanding, we provide a formal model for the evaluation and comprehension of generic sentences. 
%This model explains the puzzling flexibility in usage of generics in terms of diverse prior beliefs about properties.
%We elicit these priors experimentally and show that the resulting model predictions explain almost all of the variance in human judgments for both common and novel generics.
%This theory provides the mathematical bridge between the words we use and the concepts they describe.


\bibliographystyle{apacite}

\setlength{\bibleftmargin}{.125in}
\setlength{\bibindent}{-\bibleftmargin}

\bibliography{generics-spp}

\end{document}

