\documentclass[11pt,letterpaper]{article}

\usepackage{setspace}
%\doublespacing
\linespread{1.25}
%\linespread{2}
\usepackage{geometry}

\geometry{letterpaper, margin=1in}
%\usepackage{times}
\usepackage{pslatex}
\usepackage{apacite}
\usepackage{url}
\usepackage{graphicx}
\usepackage{caption}
\usepackage{subcaption}
\usepackage{listings}
\usepackage{color}
\usepackage{textcomp}
\usepackage{amsmath}
\usepackage{amssymb}
\usepackage{wrapfig}
\usepackage{lipsum}

\graphicspath{{figures/}}

\def\signed #1{{\leavevmode\unskip\nobreak\hfil\penalty50\hskip2em
  \hbox{}\nobreak\hfil(#1)%
  \parfillskip=0pt \finalhyphendemerits=0 \endgraf}}

\newsavebox\mybox
\newenvironment{aquote}[1]
  {\savebox\mybox{#1}\begin{quote}}
  {\signed{\usebox\mybox}\end{quote}}


 \newcommand{\denote}[1]{\mbox{ $[\![ #1 ]\!]$}}

\definecolor{Red}{RGB}{255,0,0}
\newcommand{\red}[1]{\textcolor{Red}{#1}}  
\definecolor{Green}{RGB}{10,200,100}
\definecolor{Blue}{RGB}{10,100,200}
\newcommand{\ndg}[1]{\textcolor{Green}{[ndg: #1]}}  
\newcommand{\mht}[1]{\textcolor{Blue}{[mht: #1]}}  

\usepackage{titlesec}

\setcounter{secnumdepth}{4}

\titleformat{\paragraph}
{\normalfont\normalsize\bfseries}{\theparagraph}{1em}{}
\titlespacing*{\paragraph}
{0pt}{3.25ex plus 1ex minus .2ex}{1.5ex plus .2ex}

\title{Talking in generalizations}
\author{{\large \bf Michael Henry Tessler} and {\large \bf Noah D. Goodman} \\
\{mtessler, ngoodman\} @stanford.edu\\
  Department of Psychology, Stanford University}

\begin{document}

\maketitle

Learning and development require solving problems of induction.
These are hard problems though they can be solved by adults and young children by learning from observations \cite{Markman1989} 
and in pedagogically enriched scenarios \emph{(CITE SOME PEDAGOGY?}).
%At about age of the two, a new tool for communication and learning becomes available: language.
Generalizations may also be conveyed with language.
Generic language (e.g.~\emph{Swans are white.}) provides a simple and ubiquitous way to communicate generalizations \cite{Carlson1977, Leslie2008}. 
Generic language is ubiquitous in everyday conversation and in child-directed speech \cite{Gelman2008} and children as young as two or three understand that generics refer to categories and support generalization \cite{Cimpian2008}.
%Generalizations are central to human understanding and thus, it is not surprising that language provides . 
%Generics are ubiquitous in everyday conversation as well as in child-directed speech 
Learning through generic statements is thought to be essential to the growth of conceptual knowledge \cite{Gelman2004} and how kinds are represented in the mind \cite{Leslie2008}, and yet a formal account of the meaning of generic sentences remains elusive. 

The primary hurdle in formalizing generic language is in determine what makes a generic sentence true or false (i.e. the truth conditions).
%Generics look a lot like quantifier statements (e.g. \emph{some, all, most}).
At first blush, generics look a lot like universally quantified statements as in \emph{All swans are white}, and yet generics---unlike universals---admit exceptions (e.g. there are black swans). 
Interpreting the generic as meaning ``most'' (i.e. \emph{Most swans are white}) captures many cases but fails to account for others: \emph{Robins lay eggs} seems true even though only adult, female robins do; only a tiny fraction of Mosquitos carry the virus Zika yet \emph{Mosquitos carry Zika} also sounds true. 
In fact, the very notion that the felicity of the generic can be tied to how many instances have the property (in the way that quantifier statements are) violates intuitions: \emph{Robins lay eggs} even though only the females do but \emph{Robins are female} (even though only the females are) is not a reasonable utterance.

%Habitual sentences behave in an analogous way. 
%Bill may smoke a pack a day and so we would say \emph{Bill smokes}, and if Bill goes without a cigarette for an entire family vacation (thus temporarily reducing his effective rate of smoking), we will not be so easily convinced (i.e. \emph{Bill smokes} is probably still true).
%Cases like \emph{Mosquitos carry malaria} (wherein only a tiny percentage have the property) seem to parallel habitual sentences of rare actions like \emph{Susan writes novels}. Susan may only have written 3 novels in her life, but still this seems like a valid generalization to convey.

We argue that the core meaning of a generic sentence is simple but underspecified, and that general principles of pragmatic reasoning are responsible for establishing a precise meaning in context. 
Building on recent probabilistic models of language understanding, we provide a formal model for the evaluation of generic sentences. 
This model explains the puzzling flexibility in truth conditions in terms of diverse prior beliefs about properties.
In Experiment 1, we elicit these priors experimentally and show that the resulting model predictions explain almost all of the variance in human truth judgments for familiar  generic sentences.
Generalizations about categories, however, are just one kind of generalization.
Habitual sentences describe generalizations about events (e.g. \emph{Bill smokes}, \emph{Susan mows her neighbor's lawn.}; \citeNP{Carlson1977, Carlson2005, Cohen1999}). 
Experiment 2 provides more support that this theory describes communicating generalizations in general by using an almost identical paradigm to predict the truth judgments for habitual sentences.
Finally, using habitual language, we show in Experiment 3 that communicating generalizations is sensitive to top-down moderators of expectations: It is the subjective belief of future behavior (not just what has been observed objectively in the past) that matters for communicating generalizations.

\subsubsection*{A formal model of generic meaning}

Generics express a relation between a kind (e.g. \textsc{robins}) and a property (e.g. \textsc{lays eggs}). 
For a given kind $K$ (e.g.~\textsc{robins}) and property $F$ (e.g.~\textsc{lays eggs}), we refer to the probability that an object of kind $K$ has property $F$, that is $P(F\mid K)$, as the \emph{prevalence} of $F$ within $K$.\footnote{Because we aim to explain the psycholinguistics of generics, we are generally interested in the subjective probability, not the actual frequency in the world.}
Logical quantifiers can be described as conditions on prevalence (i.e.~\emph{some} is $P(F\mid K)>0$, \emph{all} is $P(F\mid K)=1$). 
Extending this, it seems the simplest meaning for generic statements would be a similar threshold on prevalence: $P(F\mid K)>\tau$ \cite{Cohen1999}. 
However, no fixed value of the threshold, $\tau$, would allow for the extreme flexibility generics exhibit (e.g. \emph{Robins lay eggs} vs. \emph{Robins are female}; \emph{Mosquitos carry malaria}).
Building on \citeA{Lassiter2013,Lassiter2015}, we posit that this threshold is not a fixed property of the language, but is established by pragmatic inference.

Using the Rational Speech-Acts framework \cite{Frank2012,Goodman2013}, 
we model a speaker $S_2$ who reasons about a pragmatic listener $L_1$; this listener is considering the prevalence of a certain property within the category.
The listener $L_1$ has uncertainty about the appropriate threshold for the generic in this context ($\tau \sim \text{Uniform}$(0,1)), and reasons about what an informative speaker $S_1$ would be likely to say. The hypothetical speaker $S_1$ in turn reasons about an idealized literal listener $L_0$, who has access to the threshold $\tau$ (i.e. $S_1$ believes $L_0$ will interpret him in exactly the way he means). 
Writing the prevalence as $x$, this leads to a set of equations:
%
\begin{eqnarray}
P_{S_{2}}(u \mid x) & \propto &  \int_{\tau} P_{L_{1}}(x , \tau \mid u) \label{eq:S2}\\
P_{L_{1}}(x , \tau \mid u) &\propto& P_{S_{1}}(u \mid x, \tau) \cdot P(x) \cdot P(\tau) \label{eq:L1}\\
P_{S_{1}}(u \mid x, \tau) &\propto&  {P_{L_{0}}(x \mid u, \tau)}^{\alpha} \label{eq:S1}\\
P_{L_{0}}(x \mid u, \tau) &\propto& {\delta_{\denote{u}(x, \tau)} P(x)}. \label{eq:L0}
\end{eqnarray}
%
We take the pragmatic speaker $S_2$ to consider two utterances: the generic, with $\denote{u}(x, \tau) := x>\tau$, or nothing (staying silent), with $\denote{u}(x, \tau) := \text{True}$.
Equation \ref{eq:S2} can then be interpreted as a model of felicity or truth judgments \cite{Degen2014, TesslerUnderReview}.
The speaker will choose to produce the habitual when the true prevalence $x$ is more likely under $L_1$'s posterior given the generic than under her prior (implied by $S_2$ ``staying silent''). 
A fully implemented version of the model can be found at \url{http://forestdb.org/models/generics.html}.

This model can be straightforwardly adopted to describe generalizations about events (e.g.~\emph{Bill smokes}), by specifying the underlying scale to be the rate with which a person does an action, or that person's \emph{propensity} of doing the action (e.g. Bill smokes cigarettes 4 times a day.) rather than the prevalence of the property. 
The rest of the model is the same.



%We measured it empirically (n=60) for a set of properties (e.g. \textsc{lays eggs, carries malaria}; 21 in total) used in our target sentences. 
 


\subsubsection*{Summary of experiments and results}


In three experiments, we test the felicity of various generic and habitual sentences, and compare the speaker model in Eq.~\ref{eq:S2} to the human truth judgments. 
Experiment 1 ($n=100$) uses generic sentences of familiar categories (e.g.~\emph{Birds lay eggs}; \emph{Birds are female}) with no further information present.
Experiment 2 ($n=150$) uses habitual sentences of novel characters (e.g.~\emph{Bill smokes cigarettes}) and presents information about the frequency with which the event has occurred in the past (e.g. Last week, Bill smoked cigarettes 3 times.). 
For both of the experiments, we measure empirically the prior distribution $P(x)$ over prevalence ($n=60$) or propensity ($n=40$), respectively.
%In both Experiment 1 and Experiment 2, we use the empirically elicited priors and the speaker $S_2$ model to predict the truth judgments of generic and habitual sentences, respectively. 
In both cases, the speaker $S_2$ model explains almost all of the variance in human endorsements of generic and habitual sentences ($r^2(30) = 0.98$ and $r^2(93) = 0.94$, respectively) while using just the prevalence \emph{within} the particular category or the just propensity for the particular person explains relatively little of the data ($r^2(30) = 0.59$ and $r^2(93) = 0.33$).

In Experiment 3, we present participants with the same stimuli as in Expt.~2 and include extra information that either enables (e.g. Yesterday, Bill bought a pack of cigarettes.) or disables (e.g. Yesterday, Bill decided to quit smoking) the event from occurring. 
We measure participants' predictions ($n=120$) about the future frequency with which the target person will do the action (e.g. ``In the next week, how many times will be smoke cigarettes?'') and a set of different participants ($n=150$) rated the felicity of the habitual sentence. 
The speaker model $S_2$ who attempts to communicate the past frequency explains almost none of the data $r^2(63)=0.02$ while the model that wants to communicate the future frequency explains much of the data $r^2(63)=0.91$.
We also find that when no additional information is present, predicted frequency perfectly tracks past frequency (and so the two models have the same predictions for the data from Expt.~2).

In sum, our model that decides if a generic sentence is a pragmatically useful way to describe a category (or if a habitual is a useful way to describe an individual's behavior) by taking into account the listener's prior beliefs about the property (or event)---how common it is and the likely prevalence (measured empirically)---explains the puzzling flexibility of truth conditions for generic and habitual language.
%We validated this model by eliciting felicity judgments for generic and habitual sentences covering diverse properties and activities (Expt.~1b, 2b).
%To our knowledge, the experiments presented here are first empirical investigations into the truth conditions of habitual sentences and the first test of a formal model of habitual language.

%However, the enabling and disabling conditions 



%The prior $P(x)$ describes the belief distribution on the prevalence of a given property (e.g. \textsc{lays eggs}) across relevant categories for the case of generic sentences and on the rate of a given event (e.g. \textsc{smoking}) across individuals for the case of habitual sentences. 
%The shape of this distribution affects model predictions, but may vary significantly among different properties.
%We measure it empirically for a diverse set of properties ($n=40$ for use in Expt.~1) and for a set of events ($n=40$ for use in Expts.~2\&3).


%\subsubsection*{Method}
%
%We recruited 60 participants over Amazon's crowd-sourcing platform Mechanical Turk (MTurk).  
%%We chose this number of participants based on intuition with similar experiments and model comparison; 
%%since this is a quantitative experiment with no planned comparisons, power analysis in not appropriate.
%Participants were restricted to those with US IP addresses and with at least a 95\% MTurk work approval rating (the same criteria apply to all experiments reported). 
%
%On each trial of the experiment, participants filled out a table where each row was an animal category and each column was a property. 
%In order to alleviate the dependence of the distribution on our animal categories of interest, half of the animal categories were self-generated by the participant; the other half were randomly sampled from a set corresponding to the generic sentences used in Expt. 1b (e.g. \textsc{robins, mosquitos}).
%Participants were asked to fill in each row with the percentage of members of each of the species that had the property (e.g. ``50\%'').
%Each participant reported on sixteen properties (see {\it SI Section A} for more details).
%We used a set of properties associated with generics of theoretical interest (twenty-one properties in total), as described above.
% 
%\subsubsection*{Data analysis and results}
 
%\subsubsection*{Discussion}



%Yet the meaning of generic language is philosophically puzzling and has resisted precise formalization.
%We explore the idea that the core meaning of a generic sentence is simple but underspecified, 
%and that general principles of pragmatic reasoning are responsible for establishing the precise meaning in context.
%Building on recent probabilistic models of language understanding, we provide a formal model for the evaluation and comprehension of generic sentences. 
%This theory provides the mathematical bridge between the words we use and the concepts they describe.


\bibliographystyle{apacite}

\setlength{\bibleftmargin}{.125in}
\setlength{\bibindent}{-\bibleftmargin}

\bibliography{generics-spp}

\end{document}

