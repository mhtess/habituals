% 
% Annual Cognitive Science Conference
% Sample LaTeX Paper -- Proceedings Format
% 

%% Change "letterpaper" in the following line to "a4paper" if you must.

% to do:

%% expt 1: test different functional forms of prior
%% expt 2: address male / female differences with tests? and talk about the contrast class in the discussion?
%% expt 3: run. model?


\documentclass[10pt,letterpaper]{article}

\usepackage{cogsci}
\usepackage{pslatex}
\usepackage{apacite}
\usepackage{url}
\usepackage{graphicx}
\usepackage{caption}
\usepackage{subcaption}
\usepackage{listings}
\usepackage{color}
\usepackage{textcomp}
\usepackage{amsmath}
\usepackage{amssymb}
\usepackage{wrapfig}
\usepackage{lipsum}

 \newcommand{\denote}[1]{\mbox{ $[\![ #1 ]\!]$}}
 
 \definecolor{Red}{RGB}{255,0,0}
\newcommand{\red}[1]{\textcolor{Red}{#1}}  
\definecolor{Green}{RGB}{10,200,100}
\definecolor{Blue}{RGB}{10,100,200}
\definecolor{DarkOrange}{RGB}{255,100,50}
\newcommand{\ndg}[1]{\textcolor{Green}{[ndg: #1]}}  
\newcommand{\mht}[1]{\textcolor{DarkOrange}{[mht: #1]}}  


\graphicspath{{figures/}}

\def\signed #1{{\leavevmode\unskip\nobreak\hfil\penalty50\hskip2em
  \hbox{}\nobreak\hfil(#1)%
  \parfillskip=0pt \finalhyphendemerits=0 \endgraf}}

\newsavebox\mybox
\newenvironment{aquote}[1]
  {\savebox\mybox{#1}\begin{quote}}
  {\signed{\usebox\mybox}\end{quote}}



\title{Communicating generalizations about events}

\author{{\large \bf Michael Henry Tessler} (mtessler@stanford.edu) \\
 {\large \bf Noah D. Goodman} (ngoodman@stanford.edu) \\
  Department of Psychology, Stanford University}


\begin{document}

\maketitle


\begin{abstract}
\ndg{the basic framing shouldn't be habituals vs generics. it should be how do habituals work to convey generalizations about events? and then the analogy with generics motivates our approach...}
Habitual sentences (e.g. \emph{Bills smokes.}) generalize an event over time and are thought to have similarly puzzling qualities as generic sentences (e.g. \emph{Dogs bark.}). 
Little empirical work has tested this relation with respect to the truth conditions of these sentences. 
We test the analogy between habitual and generic language by applying a recent formal theory of generic language to the domain of events.
In Expts.~1 \& 2, we measure the prior and explore the predictions of this model for the
``frequency'' conditions (i.e. how often a person must do an action) 
by which habituals becomes felicitous.
% the under which various habituals may be felicitous utterances. 
In Expt.~3, we harness the richness of intuitive theories of human behavior to explore more deeply the nature of ``frequency'' conditions with respect to time.
Using our computational approach, we find the analogy between generic and habitual language to be well-suited, and that the communicating generalizations about events relies upon being informative about the future.
Methodological implications from this are discussed.

%by manipulating X and show that \emph{predictive} future frequency is what drives the model predictions and human judgments. 

\textbf{Keywords:} 
events; generics; pragmatics; Bayesian data analysis; Bayesian cognitive model
\end{abstract}

\ndg{need to revise this intro with some strong perspective taking: your target audience is smart cogsci people who don't know that much about these topics (think josh t).}

If you learn that Bill smoked a cigarette exactly three times last month, would you say that \emph{Bill smokes}?
Habitual sentences like \emph{Bill smokes.} describe generalizations about events and have similarly puzzling qualities to generic sentences (e.g. \emph{Swans are white.}) which convey generalizations about categories \cite{Carlson2005}. \ndg{what puzzling qualities? don't assume readers already know about generics...}
Generic language is central to how we communicate and learn about categories \cite{Carlson1977, Gelman2004}, and has received a lot of attention from psychologists, linguists, and philosophers.
Habitual language is presumably similarly important to our intuitive theories of events, especially the activities of people.
Yet surprisingly little empirical work has looked into the basic properties of habitual sentences.

Habitual sentences are interesting because they are likely central to our theories of other people. 
Young children can use actors' behaviors to make inferences about what the actors are like more generally \cite<e.g.>{Repacholi1997, Seiver2013}.
There is evidence to suggest the habitual sentences are weaker forms of trait language and convey behaviors that are relatively enduring \cite{Gelman1999}.
And although linguists have often described habitual and generic sentences in the same breath \cite{Carlson1977, Cohen1999}, experimentalists have yet to test the correspondence between the two. 

In this paper, we elaborate a recent computational theory of generic language to derive predictions for the truth conditions of habitual sentences. 
The theory posits the semantics of a generic statement is an uncertain threshold on the degree of property prevalence (i.e. how many instances of the kind have the feature) and derives context-sensitive meanings through basic, pragmatic inference given the distribution of property prevalence (i.e. in general, what prevalences are species likely to have) \cite{TesslerUnderReview}.
We adopt the theory by adjusting the underlying degree to be the \emph{frequency of action} (i.e. how often does a person do an action) and derive predictions for felicity judgments of the corresponding habitual.
In Expt.~1, we measure \emph{a priori} beliefs about how frequently people do a diverse set of actions.
In Expt.~2, we use those priors and the pragmatic theory to make predictions about the truth conditions of habitual sentences under different frequencies of action. 
In Expt.~3, we further explore the nature of the underlying degree scale of ``subjective frequency'' with respect to time by introducing events that prevent future frequency. 

\section{Computational model}
\ndg{since generics isn't out, it is better to talk about this as: we introduce a model building on lassiter\&goodman, which is almost the same as one we have evaluated for generics.}

Habituals express a relation between an individual (e.g. \textsc{Bill}) and an event (e.g. \textsc{to smoke}), in an analogous way to how generics express a relation between a kind (e.g. \textsc{robins}) and a property (e.g. \textsc{lays eggs}) \cite{Carlson1995}. 
\citeA{TesslerUnderReview} introduced a computational model that explains the hitherto puzzling phenomena surrounding generic language (e.g. why \emph{Robins lay eggs} is felicitous while \emph{Robins are female} is not), with a simple semantics based on the property prevalence (e.g. the proportion of robins that lay eggs or are female), coupled with basic communicative pressures to be truthful and informative. 
We apply this same model to habituals, adopting as the underlying scale the frequency of the event.
 
For a given individual $I$ (e.g.~\textsc{Bill}) and an event or behavior $B$ (e.g.~\textsc{smoking}), we refer to the probability that individual $I$ will partake in beahvior $B$, that is $P(B\mid I)$, as the \emph{subjective frequency} of $B$ for $I$.
\ndg{are you sure we want to call this subjective frequency? i think this is going to garden path people, since expt 3 is specifically making the point that it isn't frequency. how about propensity, which we define as subjective probability that I will do B in the next time window?}
%It is not clear at this point whether this frequency is a retrospective (how often has Bill smoked in the past) or a prospective frequency (how often to I expect Bill to smoke in the future); for the time being, we assume time is \emph{homogenous} with respect to past and present (i.e. however often Bill has smoked in the past will be how often he smokes in the future). 
The semantics of a habitual sentence is a simple threshold on frequency $P(B\mid I)>\tau$ \cite<c.f.>{Cohen1999}.
This threshold is not a fixed property of the language, however, but is established by pragmatic inference.
%This inference depends on event and person knowledge, but otherwise follows from a general mechanism of language.
We imagine a hypothetical, pragmatic listener ($L_1$) concerned with learning the frequency of a certain behavior for a certain individual I, $x=P(B \mid I)$, who reasons about an informative speaker ($S_1$), who in turn reasons about a literal listener ($L_0$):
\begin{eqnarray}
P_{L_{1}}(x , \tau \mid u) &\propto& P_{S_{1}}(u \mid x, \tau) \cdot P(x) \cdot P(\tau) \label{eq:L1}\\
P_{S_{1}}(u \mid x, \tau) &\propto&  {P_{L_{0}}(x \mid u, \tau)}^{\lambda} \label{eq:S1}\\
P_{L_{0}}(x \mid u, \tau) &\propto& {\delta_{\denote{u}(x, \tau)} P(x)}. \label{eq:L0}
\end{eqnarray}

\ndg{since we don't use $L_1$ directly here, i'd stick $S_2$ onto the top of the above eqnarray and just start there...}
The pragmatic listener $L_1$ (Eq.~\ref{eq:L1}) is a model of interpreting habituals: Upon hearing a habitual, what frequency is a listener likely to infer?
We can now imagine a speaker $S_2$ who reasons about this type of listener: 
%
\begin{equation} 
P_{S_{2}}(u \mid x) \propto  \int_{\theta} P_{L_{1}}(x , \tau \mid u)
\label{eq:S2}
\end{equation}
%
If we take the two utterances considered by speaker $S_2$ to be the habitual or nothing (staying silent), then Equation \ref{eq:S2} can be interpreted as a model of felicity or truth judgments \cite{Degen2014, TesslerUnderReview}.
The speaker considers the thought-processes of listener $L_1$ (Eq.~\ref{eq:L1}) and decides if the habitual is a good way to describe the frequency $x$. 
$S_2$'s decision is with respect to the alternative of saying nothing. 
He will choose to produce the habitual when the true frequency $x$ is more likely under $L_1$'s posterior given the habitual than under her prior. 
Critically, speaker $S_{2}$ doesn't have a particular meaning of the habitual in mind (i.e. doesn't have access to the threshold $\tau$), but knows that $L_{1}$ will consider it, and integrates over the likely values she'll consider.

%Herein lies an interesting deviation from generic language about natural kinds: Natural kinds change over the course of biological time. 
%People, on the other hand, develop from babies to children to adults and elderly adults; People make resolutions to explicitly change their behavior and people discover things about themselves that cause them to change behavior. 
%These higher-order beliefs about the stationarity of time with respect to events will no doubt influence speakers judgments of habituals. 
%We will explore these higher-order beliefs later when we relax the \emph{homogeneity assumption}. 

\begin{figure*}[t]
\centering
  \includegraphics[width=\textwidth]{prior-scatter-insets}
  \caption{}
  \label{fig:priorScatter}
\end{figure*}
%


\section{Experiment 1: Prior elicitation}

%Experiment 1 set out to empirically verify the long-held belief that habitual sentences are analogous to generic sentences about events. 
%We used an adapted version of the computational model presented by \citeA{TesslerUnderReview} to guide the experimental design.
%
$P(x)$ in Eqs.~\ref{eq:L1} and \ref{eq:L0} specifies prior beliefs about the frequency of a specific event or behavior $x$  (e.g. \emph{smoking}).
\ndg{note: P(x) is actually a probability over probability densities, right? this might be worth specifying more clearly in the model section...}
Given that some individuals rarely or never engage in a behavior, while others do quite frequently, we would expect the prior to be a mixture distribution between these two possibilities, similar in spirit to Hurdle Models of epidemiological data \cite{hurdleModels}.
Indeed, there may be more than these two possibilities, corresponding to different kinds of participants, for instance different expected frequencies depending on gender or age. 
In this experiment we elicit the prior $P(x)$ for different events, and explore there hypotheses about the structure of these priors.

\subsection{Method}

\subsubsection{Participants}
We recruited 40 participants from Amazon's Mechanical Turk.
Participants were restricted to those with U.S. IP addresses and who had at least a 95\% work approval rating.
The experiment took on average 12 minutes and participants were compensated \$1.25 for their work.

\subsubsection{Materials}

We created thirty-one events organized into pairs or triplets from 5 different conceptual categories: food and drug (e.g. \emph{eats caviar}, \emph{eats peanut butter}), work (e.g. \emph{sells things on eBay}, \emph{sells companies}), clothing (e.g. \emph{wears a suit}, \emph{wears a bra}), entertainment (e.g. \emph{watches professional football}, \emph{watches space launches}) and hobbies (e.g. \emph{runs}, \emph{hikes}). 
Items were chosen to intuitively cover a range of likely frequencies of action, as well as to provide a minimal comparison to another item. \ndg{minimal in what sense?}

\subsubsection{Procedure}

For each event, participants were asked two questions:
\begin{enumerate}
\item X out of every 1000 \{men, women\} has \textsc{done action} before.
\item For a typical \{man, woman\} who has \textsc{done action} before, how frequently does he or she \textsc{do action}? 
\end{enumerate}

Question 1 had a response format of entering a number, and participants were free to make the comparison number larger (default: 1000) should the event be more rare than 1 out of 1000.
Question 2 had a response format of entering a number of instances out of the time period of a year (by default). Participants were free to change the time period to week, month, or 5 years. \ndg{it's not totally clear how this procedure works... reword for clarity.}

We expected there might be different beliefs about the frequency of events depending on whether the actor is male or female, so we asked about both genders. Participants answered both questions for each gender on each slide (4 questions total per slide, order of male / female randomized between-subjects), and every participant completed all 31 items.
The experiment in full can be viewed at \url{http://stanford.edu/~mtessler/habituals/experiments/priors/priors-2.html}.

\begin{figure*}[t]
\centering
  \includegraphics[width=\textwidth]{tj-scatters1}
  \caption{}
  \label{fig:tjScatters}
\end{figure*}


\subsection{Data analysis and results}

\ndg{say why we build this bayesian data analysis model, rather than just plugging into the model -- smoothing and better capturing tails??}
Question 1 elicits the proportion of people who have done an action before. 
We model this data as coming from a Beta distribution. 
Question 2 elicits the relative frequency in the past that a person has done an action before.
This was modeled by a log-normal distribution. 
Each item was modeled independently for each gender.
%
\begin{minipage}{0.5 \textwidth} \small
\begin{align*}
d_{1} &\sim \text{Beta}(\gamma_{1}, \xi_{1}) \\
\ln d_{2} &\sim \text{Gaussian}(\mu_{2}, \sigma_{2}) \\
\end{align*}
\end{minipage}
%
We implemented this model using the probabilistic programming language WebPPL \cite{dippl}, and we learned about the credible values of the parameters and the posterior predictive distributions of the data by running MCMC for 100,000 iterations, discarding the first 50,000 for burnin.
%
%The parametrized priors are a reasonable description of the prior elicitation data, though 

\ndg{what's up with "log interval time"? need to explain, or switch to something easier to understand.}
The priors elicited indeed cover a range of possible parameter values (Figure \ref{fig:priorScatter}, scatter), resulting in parametrized distributions of dramatically different shapes (insets).  
We observe a correlation in our items between the mean \% of Americans who have \textsc{done action} before (Question 1) and the mean log-interval time between actions (Question 2) ($r_{1,2} = -0.75; r^2_{1,2} = 0.55$); our items that tend to be more popular actions also tend to be more frequent actions (e.g. \textsc{wears socks}) and visa-versa (e.g. \textsc{steals cars}), though there are notable exceptions (e.g. \textsc{plays the banjo} is not popular but done frequently when done at all, as is \textsc{smokes cigarettes}; \textsc{goes to the movies} is a popular activity though not done very often). 
This diversity is relevant because the speaker model (Eq.~\ref{eq:S2}) will produce habitual sentences (e.g. \emph{Sam goes to the movies / goes to the ballet.}) contingent on the shape of the prior distribution. 

\mht{I've cut out the "posterior predictive" $r^2$ because I think it is not a reasonable metric. We should really come up with an appropriate fitness metric.}

%To assess how well these parametrized priors capture the prior elicitation data, we compare the posterior predictive distribution to the experimental data.
%We do this by discretizing the distributions as well as truncating at the endpoints 
%The posterior predictive distributions reconstruct the experimental data reasonably well for both question 1 ($r^2_{1} = 0.86$) and question 2 ($r^2_{2} = 0.68$) \mht{are there systemic deviations?}, and we use these parameterized priors going forward.

\ndg{talk about gender differences?}

\section{Experiment 2: Felicity judgments}

\subsection{Method}

\subsubsection{Participants}

We recruited 150 participants from MTurk.
We arrive at this number, we performed a Bayesian precision analysis to determine the minimum sample size necessary to reliably ensure 95\% posterior credible intervals no larger than 0.3 for a parameter whose true value is 0.5 and for which the data is a 2 alternative forced choice. This analysis revealed a minimum sample size of 50 per item; since participants only completed about one third of the items, we recruited 150 participants.
The experiment took 4 minutes on average on participants were compensated \$0.55 for their work.

\subsubsection{Procedure and materials}

On each trial, participants were presented with a \emph{past frequency statement} for a given event or behavior. 
The frequency statement was of the form: ``In the past $M$ \{weeks, months, years\}, \textsc{person} \textsc{event} 3 times''.
We kept constant the number of times the action was done (3) in order to isolate the effects of the time window. 
The particular intervals used (number $M$ and window \{weeks, months, years\}) were selected by examining the predictions of the speaker model (Eq.~\ref{eq:S2}) for each item independently, and aimed to elicit a variety of predictions.
The materials were the same as in Expt. ~1.

Participants were asked whether they agreed or disagreed with the corresponding habitual sentence: ``\textsc{person event}''.
We selected 6 items for which the priors for men and women varied substantially in Expt.~1 to explore whether or not the prior distribution in Eq.~\ref{eq:L1}, \ref{eq:L0} should be with respect to all people or to males and females separately (e.g. if there is a frequency at which a male would be judged to (habitually) \textsc{wear a bra} but a woman would not be judged to). 
The experiment in full can be viewed at \url{http://stanford.edu/~mtessler/habituals/experiments/truth-judgments/tj-2.html}.

%The model predicts that some items would not exhibit variability in endorsements in given ranges (e.g. if a person stole a car 3 times in the past \emph{week} vs. \emph{month}); we omitted particular intervals when the model predicted non-meaningful differences in endorsements. 

%\subsubsection{Materials}

\subsection{Behavioral results}

On each trial of the experiment, the participant was told a person did a particular action 3 times during some time window. 
Figure \ref{fig:tjScatters} (left) shows the correspondence between the time window given and the felicity of the corresponding habitual sentence. 
It is clear that a habitual sentence can receive strong agreement even when the actions are very infrequent (log time interval $>$ 4; time interval of subsequent actions years or more; e.g. writing 3 novels, stealing 3 cars in a 5 year interval).
We also see habitual sentences that participants are reluctant to endorse completely (e.g. wears socks, drinks coffee), even when they are relatively frequent in occurrence (spec. 3 times in a one month interval).
In our data, actions that are completed multiple times in a very short interval (spec, 3 times in a one week interval) receive at least 75\% endorsement, though there is still variability among them (e.g. smoking cigarettes, wearing socks are endorsed the least), suggesting that even actions that are completed almost everyday can be not good enough to generalize.
Overall, the time interval predicts only a fraction of the variability in responses ($r^2(93) = 0.33$).
For actions that are done on the time scale of years or more (upper median of time interval), time interval itself no longer explains the endorsements  ($r^2(50) = 0.07$)

We notice that none of our items receive less than 25\% endorsement (i.e. only 75\% of participants disagree with the felicity of the utterance).
This may be due to the fact that actor has done the action in the past a plurality of times; we would expect to get strong disagreement with the utterance (endorsement = 0\%) when the person has never done the action, or perhaps done it only once.

\subsection{Model fit and results}

We used the pragmatic speaker model $S_2$ (Eq.~\ref{eq:S2}) with the posterior predictive of the prior data (Expt.~1) as $P(x)$  to predict felicity judgments in Expt.~2.
We modeled $P(x)$ as a mixture of individuals with the possibility of carrying out the action and those without the possibility of doing it. 
The posterior predictive of the prior data was constructed by forward-sampling the mixture component $\theta$ (determined by Q1: number of people who had done the action before, see Expt.~1 data analysis).
If the sampled person was the kind of person to have done the action before, the frequency was sampled a likely frequency of action (determined by Q2). 
If they were not the type of person to have done the action, we assume they will never or only rarely do it
\footnote{These assumptions, while probably valid if a learner knows nothing else about the person, will be challenged in Expt.~3}.

\begin{minipage}{0.5 \textwidth} \small
\begin{align*}
\theta & \sim \text{Beta}(\gamma_{1}, \xi_{1}) \\ 
\ln x & \sim \begin{cases} 
		\text{Gaussian}(\mu_{2}, \sigma_{2}) &\mbox{if } \text{Bernoulli}(\theta) = \textsc{t} \\
				\delta_{x=\infty} &\mbox{if } \text{Bernoulli}(\theta) = \textsc{f} \\
		\end{cases} \\
\end{align*}
\end{minipage}

Having fit the prior empirically, the model was one parameter -- the speaker optimality parameter in Eq.~\ref{eq:S1}. 
Additionally, we include a data analytic parameter to model random guessing behavior; this provides a rough measure of how much variance is unexplained by the pragmatics model. 
We use Bayesian data analytic techniques to integrate out these parameters \cite{LW2014}, and compare the posterior predictive distribution of this model to the empirical data in Expt.~2.

To attain credible values of the model parameters as well as the posterior predictive distribution over responses, we collected 3 MCMC chains of 30,000 iterations, discarding the first 15,000 iterations of each chain for burn in.
The Maximum A-Posteriori (MAP) value and 95\% highest probability density (HPD) interval for the speaker optimality parameter in Eq.~\ref{eq:S1} is 4.7 [3.7,5.2].
The MAP and HPD interval for the data-analytic guessing parameter is 0.004 [0.0003, 0.03], suggesting that there is not a substantial amount of the data that is better explained by a model of random guessing than by our pragmatic speaker model.

The probabilistic pragmatics model does a better job of accounting for the variability in responses ($r^2(93) = 0.89$), including actions done on the time scale of years or more  ($r^2(50) = 0.89$) (Figure \ref{fig:tjScatters} right).
The model designed to describe generic language (e.g. \emph{Swans are white.}) can describe habitual language (e.g. \emph{John smokes.}) equally as well by specifying the underlying degree as the frequency with which an individuals performs an action.
This provides a formal bridge between generalizations of properties about categories (i.e. \emph{generics}) and generalizations about individuals or events (i.e. \emph{habituals}), a connection often noted in the linguistics literature \cite{Carlson1977, Carlson2005, Cohen1999}. 

\section{Experiment 3: Time and subjective frequency}

People change in a way that natural kinds cannot: People can modify their behavior overnight.
Yet, there is an ambiguity in the frequency degree semantics over which the pragmatic theory operates.
In some sense, one can only know \emph{past frequency} of action because that is all that has occurred.
On the other hand, language has a communicative function, and only \emph{predictive frequency} will be helpful in the future.
In Expt.~3, we explore how knowledge of certain causal factors (e.g. the decision to retire; the development of an allergy) interacts with the past frequency of action to modulate the felicity of habitual sentences (e.g. \emph{John writes novels.}; \emph{John eats peanut butter.}).

\subsection{Method}
\begin{figure*}[t]
\centering
  \includegraphics[width=\textwidth]{truth-judgments-3items-withtj2.pdf}
  \caption{}
  \label{fig:tj3}
\end{figure*}

\subsubsection{Participants} 

We recruited 150 participants from MTurk, using the same criterion as Expt.~2.
The experiment took 4 minutes on average on participants were compensated \$0.40 for their work.

\subsubsection{Procedure}

The procedure was identical to Expt.~2 except for the inclusion of a second sentence on a subset of trials. 
On all trials, participants were presented with a \emph{past frequency sentence} (see Expt.~2).
Additionally, on one third of the trials, participants were presented with a \textbf{preventative sentence} (e.g. \emph{Yesterday, Bill quit smoking.}). %that aimed to decrease the acceptability of the habitual sentence (\emph{Bill smokes}).
% even in the face of strong past frequency evidence (\emph{Last week, Bill smoked 3 times.}).
On one third of the trials, participants were presented with an \textbf{enabling sentence} (\emph{Yesterday, Bill bought a pack of cigarettes.}) %that sought to reinforce the past frequency that aimed to increase the acceptability of the habitual sentence when the past frequency evidence was not strong. 
The final third of trials had no additional evidence and were identical to Expt.~2. 

\subsubsection{Materials}

Twenty-one of the original thirty-one items were used in order to shorten the experiment.
To increase expected variability, participants saw the frequencies that led to most intermediate endorsement of the habitual in Expt.~2. 
In addition, we did not include trials for both male and female names for the select items we did in Expt.~2, since we saw no differences in their endorsements of the habitual.
The dependent measure was the same as in Expt.~2. 
The experiment in full can be viewed at \url{http://stanford.edu/~mtessler/habituals/experiments/truth-judgments/tj-3-preventative.html}.

\subsection{Behavioral results}

Figure \ref{fig:tj3} shows the results of the experiment, together with the analogous data from Expt.~2. 
There is a clear and consistent negative effect of preventative information on the tendency to endorse the habitual sentence (purple bars).
When collapsing across items and subjecting the data to a mixed-effects model with random by-participant effects of intercept and random by-item effects of intercept and conditions, we find evidence for a small effect of \emph{enabling} conditions on endorsements (M =  0.89 [0.88, 0.91]) as compared to baseline (M = 0.85 [0.83, 0.87]) --- $\beta = 0.42; SE = 0.15; z = 2.8; p = 0.005$, in addition to a large effect of \emph{preventative} conditions on endorsements (M = 0.29 [0.26, 0.31]) --- $\beta = -3.22; SE = 0.21; z = -15.2; p < 0.001$. 
Finally, we observe endorsements in this experiment that are appreciably higher than in Expt.~2 for the same items (compare green bars to red bars).
This may be due, in part, to an effect of the experiment context on participants: Participants are viewing extra information that strongly disables the action from occurring; when that information is not present, participants may be more inclined to endorse the habitual.

%     condition      mean  ci_lower  ci_upper
%        (fctr)     (dbl)     (dbl)     (dbl)
%1     enabling 0.8942857 0.8752381 0.9133333
%2     baseline 0.8504762 0.8304762 0.8714524
%3 preventative 0.2885714 0.2628333 0.3161905


%Generalized linear mixed model fit by maximum likelihood (Laplace Approximation) ['glmerMod']
% Family: binomial  ( logit )
%Formula: response ~ condition + (1 | workerid) + (1 + condition | habitual)
%   Data: d
%
%     AIC      BIC   logLik deviance df.resid 
%  2684.2   2744.7  -1332.1   2664.2     3140 
%
%Scaled residuals: 
%    Min      1Q  Median      3Q     Max 
%-3.8931 -0.4064 -0.2551  0.4292  7.4117 
%
%Random effects:
% Groups   Name                  Variance Std.Dev. Corr       
% workerid (Intercept)           0.891215 0.9440              
% habitual (Intercept)           0.316971 0.5630              
%          conditionenabling     0.004135 0.0643    0.09      
%          conditionpreventative 0.561694 0.7495   -0.61  0.73
%Number of obs: 3150, groups:  workerid, 150; habitual, 21
%
%Fixed effects:
%                      Estimate Std. Error z value Pr(>|z|)    
%(Intercept)            -2.1220     0.1791 -11.848  < 2e-16 ***
%conditionenabling      -0.4190     0.1508  -2.779  0.00546 ** 
%conditionpreventative   3.2262     0.2125  15.181  < 2e-16 ***
%---

These results suggest that the felicity of habituals is based on an underlying scale of \emph{predictive} frequency of action.
Habitual language is useful when it corresponds to future predictions of people's actions.

%\subsection{Model results}


\section{Discussion}

To our knowledge, this is the first empirical evidence addressing the analogy between habitual sentences and generic sentences.
We are able to do this using a formal theory of the semantics and pragmatics of generic language.
We adopt as the underlying degree scale the \emph{subjective frequency} with which a person does an action. 
In Expt.~3, we showed that the notion of subjective frequency is tied to the future, \emph{predictive} frequency, which depends upon both past frequency and other conceptual factors.

\mht{kind of rough and needs more flowers}
In the psychological literature, habituals have received relatively little attention as compared with \emph{generics}. 
Generics often use a bare plural (e.g. \emph{Bears like to eat ants.}) and don't lay claim to any well-defined set of individuals (e.g. many bears may not like to eat ants).
Habituals use the simple present tense (e.g. \emph{John smokes}) without any well-defined period of time (e.g. John may go many days without smoking). 
In both cases, we are able to explain the varying truth conditions of these kinds of sentences by using a pragmatically inferred threshold on the set of individuals or on the period of time, respectively. 
Scales, and scalar representations, provide a simple and general quantitative way to express truth conditions.

Generics are ubiquitous in everyday conversation and in child-directed speech \cite{Gelman2008} and are thought to be central to how concepts are developed \cite{Gelman2004}.
What might habitual language be used for?
One possibility is that we describe individuals in terms of their enduring traits and habits. 
Habituals may be weaker forms of trait language \cite{Gelman1999} and perhaps may be the bridge between observations of particular instances in time and enduring, essentialist beliefs about people.
Future work should explore this connection more directly.
	
\mht{kind of rough and needs more meat}
The similarities between habitual and generic language may be useful for methodological reasons. 
In Expt.~3, we manipulated participants' beliefs about the expected, future frequency of a behavior in a person.
This showed that the underlying degree scale is likely to be \emph{predictive} frequency, as opposed to past frequency. 
We were able to do this because people have rich, structured theories of people's behavior that varies over time. 
In general, the different aspects of intuitive theories of people vs. categories may facilitate further exploration of how beliefs and language play off one another.


%The habitual sentences we explored were all presented in the simple, present tense.
%Generalizations can also occur in the past tense (e.g. \emph{Bill smoked}, or \emph{Bill used to smoke.}) and in the future tense (e.g. \emph{Bill will smoke}). 
%The present tense is interesting because it is \emph{a priori} unclear when, in time, the sentence is making reference to.  
%
%\begin{enumerate}
%\item Methodological implications (using habituals as opposed to generics)
%\item Contrast class finding (men vs women)
%\item Tense (past vs future vs present)
%\end{enumerate}


%people do things and infer what they \emph{are like} (e.g. if they often do this thing). 
%Understanding habitual language can shed light on a person's intuitive theories of other people, which might in turn be the basis of how we learn intuitive theories of groups (or, stereotypes). 
%Additionally, because of people's intuitive theories of others are so richly structured according to events (e.g. having hobbies, having a job, eating certain foods and not others; or more generally, being able to complete certain actions and not others),
%habitual language may give us insight into those theories. \red{eek}


\bibliographystyle{apacite}

\setlength{\bibleftmargin}{.125in}
\setlength{\bibindent}{-\bibleftmargin}

\bibliography{habituals-cogsci2016}


\end{document}
