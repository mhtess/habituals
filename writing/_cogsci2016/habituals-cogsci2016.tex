% 
% Annual Cognitive Science Conference
% Sample LaTeX Paper -- Proceedings Format
% 

%% Change "letterpaper" in the following line to "a4paper" if you must.

\documentclass[10pt,letterpaper]{article}

\usepackage{cogsci}
\usepackage{pslatex}
\usepackage{apacite}
\usepackage{url}
\usepackage{graphicx}
\usepackage{caption}
\usepackage{subcaption}
\usepackage{listings}
\usepackage{color}
\usepackage{textcomp}
\usepackage{amsmath}
\usepackage{amssymb}
\usepackage{wrapfig}
\usepackage{lipsum}

\graphicspath{{figures/}}

\def\signed #1{{\leavevmode\unskip\nobreak\hfil\penalty50\hskip2em
  \hbox{}\nobreak\hfil(#1)%
  \parfillskip=0pt \finalhyphendemerits=0 \endgraf}}

\newsavebox\mybox
\newenvironment{aquote}[1]
  {\savebox\mybox{#1}\begin{quote}}
  {\signed{\usebox\mybox}\end{quote}}


\definecolor{Red}{RGB}{255,0,0}
\newcommand{\red}[1]{\textcolor{Red}{#1}}  


\title{Communicating generalizations about events}

\author{{\large \bf Michael Henry Tessler} (mtessler@stanford.edu) \\ {\large \bf Noah D. Goodman} (ngoodman@stanford.edu) \\
  Department of Psychology, Stanford University}


\begin{document}

\maketitle


\begin{abstract}
Habitual sentences (e.g. \emph{Bills smokes}) convey generalizations about events and have similarly puzzling qualities as generic sentences (e.g. \emph{Dogs bark.}). 
Surprisingly little empirical work has examined the conditions by which a speaker may assert a habitual utterance and the inferences listeners derive from such utterances.
We take the analogy between generics and habituals seriously by adapting a recent pragmatic theory of generic language to the domain of events.
In Expt.~1, we explore the predictions of this model for the ``frequency'' conditions (i.e. how often a person must do an action) under which various event generalizations may be felicitous utterances. 
In Expt.~2, we explore the nature of ``frequency'' conditions by manipulating X and show that \emph{predictive} future frequency is what drives the model predictions and human judgments. 

\textbf{Keywords:} 
events; generics; pragmatics
\end{abstract}

\section{Introduction}

Generics sentences (e.g. \emph{Swans are white.}) describe generalizations about categories and are believed to be central to how we communicate and learn about categories \cite{Carlson1977, Gelman2004}.
Habitual sentences (e.g. \emph{Bill smokes.}) make generalizations about events and believed to behave like generics \cite{Carlson2005}.

Generic sentences have received a lot of attention from psychologists, linguists, and philosophers and yet surprisingly little empirical work has looked into the basic inferences that are made from habitual sentences.

A recent theory of generic language posits a simple, basic meaning of a generic based on the property prevalence (i.e. how many instances of the kind have the feature) and derives context-sensitive meanings through basic communicative principles \cite{TesslerUnderReview}.

\section{Computational model}

\section{Experiment 1}

\section{Experiment 2}

\section{Discussion}

\bibliographystyle{apacite}

\setlength{\bibleftmargin}{.125in}
\setlength{\bibindent}{-\bibleftmargin}

\bibliography{habituals-cogsci2016}


\end{document}
